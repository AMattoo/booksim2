\documentclass[11pt]{article}
\usepackage{fancyhdr}
\usepackage[dvips]{graphicx} 
\usepackage{amsmath,amssymb} 
\usepackage{epsfig}
\usepackage{calc}

\newcommand{\simname}{BookSim~}

%%%%%%%%%%%%%%%%%%%%%%%%%%%%%%%%%%%%%%%%%%%%%%%%%%%%%%%%%%%%%%%%%%%%%%

% Setup the margin sizes.

\evensidemargin = 0in
\oddsidemargin = 0in
\textwidth = 6.5in

\topmargin = -0.5in
\textheight = 9in

%%%%%%%%%%%%%%%%%%%%%%%%%%%%%%%%%%%%%%%%%%%%%%%%%%%%%%%%%%%%%%%%%%%%%%

\author{Ted Jiang, George Michelogiannakis, Daniel Becker, Brian Towles and William J. Dally}
\title{\simname 2.0 User's Guide}

\begin{document}

\maketitle
\tableofcontents

\pagestyle{fancy}
%\renewcommand{\chaptermark}[1]{\markboth{#1}{}}
\renewcommand{\sectionmark}[1]{\markright{\thesection\ #1}}
\fancyhf{} % delete current setting for header and footer
\fancyhead[LE,RO]{\bfseries\thepage}
\fancyhead[LO]{\bfseries\rightmark}
\fancyhead[RE]{\bfseries\leftmark}
\renewcommand{\headrulewidth}{0.5pt}
\renewcommand{\footrulewidth}{0.5pt}
\addtolength{\headheight}{0.5pt} % make space for the rule
\cfoot{\small\today}
\fancypagestyle{plain}{%
    \fancyhf{} % get rid of headers on plain pages
    \renewcommand{\headrulewidth}{0pt} % and the line
    \renewcommand{\footrulewidth}{0pt} % and the line
} 


%%%%%%%%%%%%%%%%%%%%%%%%%%%%%%%%%%%%%%%%%%%%%%%%%%%%%%%%%%%%%%%%%%%%%%

\newenvironment{opt_list}[1]{\begin{list}{}{\renewcommand{\makelabel}[1]%
{\texttt{##1}\hfil}\settowidth{\labelwidth}{\texttt{#1}}\setlength{\leftmargin}%
{\labelwidth+\labelsep}}}{\end{list}}

\section{Introduction}

This document describes the use of the \simname interconnection
network simulator.  The simulator is designed as a companion to the
textbook ``Principles and Practices of Interconnection Networks''
(PPIN) published by Morgan Kaufmann (ISBN: 0122007514) and it is
assumed that is reader is familiar with the material covered in that
text.

This user guide is fairly brief as, with most simulators, the best way
to learn and {\it understand} the simulator is to study the code.
Most of the simulator's components are designed to be modular so tasks
such as adding a new routing algorithm, topology, or router
microarchitecture should not require a complete redesign of the code.
Once you have downloaded the code, compiled it, and run a simple
example (Section~\ref{sec:get_started}), the more detailed examples of
Section~\ref{sec:examples} give a good overview of the capabilities of
the simulator.  A list of configuration options is provided in
Section~\ref{sec:config_params} for reference.

\section{Getting started}
\label{sec:get_started}

\subsection{Downloading and building the simulator}
\label{sec:download}

The latest version of the simulator is available from
\texttt{http://cva.stanford.edu} as a compressed tar archive.  UNIX/Linux
users can extract this archive using the tar utility
\begin{verbatim}
  tar xvfz booksim-1.0.tar.gz  
\end{verbatim}
Windows users can use a compression program such as WinZip to extract
the archive.

The simulator itself is written in C++ and has been specifically
tested with GNU's G++ compiler (version $\ge3$).  In addition, both a
LEX and YACC tool (also known as FLEX and BISON) are needed to create
the configuration parser.  These are standard tools in any UNIX/Linux
development environment.  It is suggested that Windows users download
the CYGWIN versions (\texttt{http://www.cygwin.com}) of these UNIX
development tools to simplify their compilation process.  The
\texttt{Makefile} should be edited so that the first lines give the
paths to the tools.  At Stanford, for example, the compiler, YACC, and
LEX are stored in the \texttt{/usr/pubsw/bin} directory.  The
\texttt{Makefile} reflects this:
\begin{verbatim}
CPP    = /usr/pubsw/bin/g++
YACC   = /usr/pubsw/bin/byacc -d
LEX    = /usr/pubsw/bin/flex
\end{verbatim}
Then, the simulator can be compiled by running \texttt{make} in the
directory that contains the \texttt{Makefile}.

\subsection{Running a simulation}
\label{sec:run_example}

The syntax of the simulator is simply
\begin{verbatim}
  booksim [configfile]
\end{verbatim}
The optional parameter \texttt{configfile} is a file that contains
configuration information for the simulator.  So, for example, to
simulate the performance of a simple $8 \times 8$ torus (8-ary 2-cube)
network on uniform traffic, a configuration such as the one shown in
Figure~\ref{fig:config_example} could be used.  This particular
configuration is stored in \texttt{examples/torus88}.

\begin{figure}
\begin{verbatim}
  // Topology
  topology = torus;
  k        = 8;
  n        = 2;

  // Routing
  routing_function = dim_order;

  // Flow control
  num_vcs = 2;

  // Traffic
  traffic        = uniform;
  injection_rate = 0.15;
\end{verbatim}
\caption{Example configuration file for simulating a 8-ary 2-cube
network.}
\label{fig:config_example}
\end{figure}

In addition to specifying the topology, the configuration file also
contains basic information about the routing algorithm, flow control,
and traffic.  This simple example uses dimension-order routing and, to
ensure deadlock-freedom of this routing function in the torus, two
virtual channels are required.  The \texttt{injection\_rate} parameter
is added to tell the simulator to inject 0.15 flits per simulation
cycle per node.  Because the simulator operates at the flit level,
most parameters are specified in units of flits as is the case with
the \texttt{injection\_rate}.  Also, any line of the configuration
that begins with \texttt{//} is treated as a comment and ignored by
the simulator.  A detailed list of configuration parameters is given in
Section~\ref{sec:config_params}.

\subsection{Simulation output}

Continuing our example, running the torus simulation produces the
output shown in Figure~\ref{fig:sim_output}.  Each simulation has
three basic phases: warm up, measurement, and drain.  The length of
the warm up and measurement phases is a multiple of a basic sample
period (defined by \texttt{sample\_period} in the configuration).  As
shown in the figure, the current latency and throughput (rate of
accepted packets) for the simulation is printed after each sample
period.  The overall throughput is determined by the lowest throughput
of all the destination in the network, but the average throughput is
also displayed.

\begin{figure}
\begin{verbatim}
%=================================
% Average latency = 6.02008
% Accepted packets = 0.11 at node 52 (avg = 0.147094)
% latency change    = 1
% throughput change = 1

...

% Warmed up ...
%=================================
% Average latency = 6.0796
% Accepted packets = 0.119 at node 5 (avg = 0.148266)
% latency change    = 0.00562457
% throughput change = 0.00379387

...

% Draining all recorded packets ...
% Draining remaining packets ...
====== Traffic class 0 ======
Overall average latency = 6.09083 (1 samples)
Overall average accepted rate = 0.149475 (1 samples)
Overall min accepted rate = 0.138551 (1 samples)
\end{verbatim}
\caption{Simulator output from running the \texttt{examples/torus88}
configuration file.}
\label{fig:sim_output}
\end{figure}

After the warm up periods have passed, the simulator prints the
``\texttt{Warmed up}'' message and resets all the simulation statistics.
Then, the measurement phase begins and statistics continue to be
reported after each sample period.  Once the measurement periods have
passed, all the measurement packets are drained from the network
before final latency and throughput numbers are reported.  Details of
the configuration parameters used to control the length of the
simulation phases are covered in Section~\ref{sec:sim_params}.

\section{Examples}
\label{sec:examples}

One of the most basic performance measures of any interconnection
network is its latency versus offered load.
Figure~\ref{fig:lat_vs_load} shows a simple configuration file for
making this measurement in a 8-ary 2-mesh network under the transpose
traffic pattern.  This configuration was used to generate Figure 25.2
in PPIN.  The particular configuration accounts for some small delays
and pipelining of the input-queued router and also introduces a small
input speedup to account for any inefficiencies in allocation.  By
running simulations for many increments of \texttt{injection\_rate},
the average latency curve can be found.  Then, to compare the
performance of dimension-order routing against several other routing
algorithms, for example, the \texttt{routing\_function} option can be
changed.

\begin{figure}
\begin{verbatim}
// Topology

topology = mesh;
k = 8;
n = 2;

// Routing

routing_function = dim_order;

// Flow control

num_vcs     = 8;
vc_buf_size = 8;

wait_for_tail_credit = 1;

// Router architecture

vc_allocator = islip;
sw_allocator = islip;
alloc_iters  = 1;

credit_delay   = 2;
routing_delay  = 1;
vc_alloc_delay = 1;

input_speedup     = 2;
output_speedup    = 1;
internal_speedup  = 1.0;

// Traffic

traffic                = transpose;
const_flits_per_packet = 20;

// Simulation

sim_type       = latency;
injection_rate = 0.1;
\end{verbatim}
\caption{A typical configuration file (\texttt{examples/mesh88\_lat})
for creating a latency versus offered load curve for a 8-ary 2-mesh
network.}
\label{fig:lat_vs_load}
\end{figure}

Figure~\ref{fig:fly_dist} shows a configuration file that can be used
to determine the distribution of packet latencies in a 2-ary 6-fly
network that uses age-based arbitration.  Note the use of the
\texttt{priority} configuration parameter along with the
\texttt{select} allocators that account for packet priorities.  The
simulator does not output latency distributions by default, but by
editing \texttt{trafficmanager.cpp}, setting the configuration
variable \texttt{DISPLAY\_LAT\_DIST} to true, and recompiling, the
distribution will be displayed at the end of the simulation.  This
technique was used to produced the distribution shown in Figure 25.12
of PPIN.

\begin{figure}
\begin{verbatim}
// Topology

topology = fly;
k = 2;
n = 6;

// Routing

routing_function = dest_tag;

// Flow control

num_vcs     = 8;
vc_buf_size = 8;

wait_for_tail_credit = 1;

// Router architecture

vc_allocator = select;
sw_allocator = select;
alloc_iters  = 1;

credit_delay   = 2;
routing_delay  = 1;
vc_alloc_delay = 1;

input_speedup     = 2;
output_speedup    = 1;
internal_speedup  = 1.0;

// Traffic

traffic                = uniform;
const_flits_per_packet = 20;
priority               = age;

// Simulation

sim_type       = latency;
injection_rate = 0.1;
\end{verbatim}
\caption{A configuration file (\texttt{examples/fly26\_age}) for
finding the distribution of packet latencies using age-based
arbitration.}
\label{fig:fly_dist}
\end{figure}

As a final example, Figure~\ref{fig:single} shows the use of the
special single-node topology to test the performance of a switch
allocator --- in this case, the iSLIP allocator.  The
\texttt{in\_ports} and \texttt{out\_ports} options set up a simulation
of an $8\times 8$ crossbar.

\begin{figure}
\begin{verbatim}
// Topology

topology  = single;
in_ports  = 8;
out_ports = 8;

// Routing

routing_function = single;

// Flow control

vc_allocator = islip;
sw_allocator = islip;
alloc_iters  = 2;

num_vcs     = 8;
vc_buf_size = 1000;

wait_for_tail_credit = 0;

// Simulation

sim_type       = latency;
injection_rate = 0.1;
\end{verbatim}
\caption{A single-node configuration file (\texttt{examples/single})
for testing the performance of a switch allocator.}
\label{fig:single}
\end{figure}

\section{Configuration parameters}
\label{sec:config_params}

All information used to configure a simulation is passed through a
configuration file as illustrated by the example in
Section~\ref{sec:run_example}.  This section lists the existing
configuration parameters --- a user can incorporate additional options
by changing the \texttt{booksim\_config.cpp} file.

\subsection{Topologies}
\label{sec:topos}

The \texttt{topology} parameter determines the underlying topology of the
network and the simulator supports four basic topologies:
\begin{opt_list}{single}
\item[fly] A $k$-ary $n$-fly (butterfly) topology. The \texttt{k}
parameter determines the network's radix and the \texttt{n} parameter
determines the network's dimension.

\item[mesh] A $k$-ary $n$-mesh (mesh) topology. The \texttt{k}
parameter determines the network's radix and the \texttt{n} parameter determines
the network's dimension.

\item[single] A network with a single node, used for testing single
router performance.  The number of input and output ports for the node
is determined by the \texttt{in\_ports} and \texttt{out\_ports} parameters,
respectively.

\item[torus] A $k$-ary $n$-cube (torus) topology.  The \texttt{k}
parameter determines the network's radix and the \texttt{n} parameter determines
the network's dimension.
\end{opt_list}

Both the \texttt{mesh} and \texttt{torus} topologies support the
addition of random link failures with the \texttt{link\_failures}
parameter.  The value of \texttt{link\_failures} determines the number
of channels that are randomly removed from the topology and are thus
no longer available for forwarding packets.  Moreover, the
randomization for failed channels is controlled by selecting an
integer value for the \texttt{fail\_seed} parameter --- a fixed seed
gives a fixed set of failed channels, independent of other
randomization in the simulation.  Also, note that only certain routing
functions support this feature (see Section~\ref{sec:routing_algs}).

\subsection{Physical sub-networks}
\label{sec:physical_subnets}

The \texttt{physical\_subnetworks} parameter defines the number of physical sub-networks present in the network (defaults to one).
All sub-networks receive the same configuration parameters and thus are identical. Traffic sources
maintain an injection queue for each sub-network. The packet generation process is unaffected. It enqueues generated packets into the
proper sub-network queue according to a division function in the traffic manager. At every cycle, flits at the head of each queue attempt to be injected.
Traffic destinations can eject one flit from each sub-network each cycle.


\subsection{Routing algorithms}
\label{sec:routing_algs}

The \texttt{routing\_function} parameter selects a routing algorithm
for the topology.  Many routing algorithms need multiple virtual
channels for deadlock freedom (VCDF).

\begin{opt_list}{dim\_order\_bal}

\item[dim\_order] Dimension-order routing.  Works for the
\texttt{mesh} topology (1 VCDF) and for the \texttt{torus} topology (2
VCDF).

\item[dim\_order\_bal] Dimension-order routing for the
\texttt{torus} topology with a more balanced use of VCs to
avoid deadlock (2 VCDF).

\item[dim\_order\_ni] A non-interfering version of
dimension-order routing.  Works on the \texttt{torus} or \texttt{mesh}
topology and requires one VC per network terminal.

\item[min\_adapt] A minimal adaptive routing algorithm for
the \texttt{mesh} topology (2 VCDF) and for the \texttt{torus}
topology (3 VCDF).

\item[planar\_adapt] Planar-adaptive routing for the
\texttt{mesh} topology (2 VCDF).  Supports routing around failed channels.

\item[romm] ROMM routing for the \texttt{mesh} (2 VCDF).
Load is balanced by routing in two phases: one from the source to a
random intermediate node in the minimal quadrant and a second from the
intermediate to the destination.

\item[romm\_ni] A non-interfering version of ROMM routing for
the \texttt{mesh} that requires one VC per network terminal.

\item[single] A dummy routing function used for the
\texttt{single} topology.

\item[valiant] Valiant's randomized routing algorithm for the
\texttt{mesh} (2 VCDF) and \texttt{torus} (4 VCDF) topology.

\item[valiant\_ni] A non-interfering version of Valiant's algorithm
for the \texttt{torus} that requires 4 VCs per network terminal.

\end{opt_list}

Also, the simulator code is structured so that additional routing
algorithms can be added with minimal changes to the overall simulator
(see the \texttt{routefunc.cpp} file in the simulator's source code).

\subsection{Flow control}

The simulator supports basic virtual-channel flow control with
credit-based backpressure.  

\begin{opt_list}{wait\_for\_tail\_credit}

\item[num\_vcs] The number of virtual channels per physical channel.

\item[vc\_buf\_size] The depth of each virtual in flits.

\item[voq] If non-zero, use virtual-output queuing.  With virtual
output queuing, a separate virtual channel is assigned to each
destination in the network.  This option is most useful when used with
a non-interfering routing algorithm (Section~\ref{sec:routing_algs}).
  
\item[wait\_for\_tail\_credit] If non-zero, do not reallocate a virtual
channel until the tail flit has left that virtual channel.  This
conservative approach prevents a dependency from being formed between
two packets sharing the same virtual channel in succession.
\end{opt_list}

\subsection{Router organizations}

The simulator also supports two different router microarchitectures.
The input-queued router follows the general organization described in
PPIN while the event-driven router is modeled after the router used in
the Avici TSR and described in U.S. Patent 6,370,145.  The
microarchitecture is selected using the \texttt{router} option.  Also,
both routers share a small set of options.

\begin{opt_list}{internal\_speedup}
\item[credit\_delay] The processing delay (in cycles) for a credit.
Does not include the wire delay for transmitting the credit.

\item[internal\_speedup] An arbitrary speedup of the internals of the
routers over the channel transmission rate.  For example, a speedup
1.5 means that, on average, 1.5 flits can be forwarded by the router
in the time required for a single flit to be transmitted across a
channel.  Also, the configuration parser expects a floating point
number for this field, so integer speedups should also include a
decimal point (e.g. ``2.0'').

\item[output\_delay] The processing delay incurred in the output queue
of a router.
\end{opt_list}

\subsubsection{The input-queued router}
\label{sec:iq_router}

The input-queued router (\texttt{router = iq}) follows the pipeline
described in PPIN of route computation, virtual-channel allocation,
switch allocation, and switch traversal.  There are several options
specific to the input-queued router.

\begin{opt_list}{st\_prepare\_delay}

\item[input\_speedup] An integer speedup of the input ports in space.
A speedup of 2, for example, gives each input two input ports into the
crossbar.  Access to these ports is statically allocated based on the
virtual channel number: virtual channel $v$ at input $i$ is connected
to port $i \cdot s + (v \mod s)$ for an input speedup of $s$.

\item[output\_speedup] An integer speedup of the output ports in
space.  Similar to \texttt{input\_speedup}

\item[routing\_delay] The delay (in cycles) of route computation.

\item[sw\_allocator] The type of allocator used for switch allocation.
See Section~\ref{sec:alloc} for a list of the possible allocators.

\item[sw\_alloc\_delay] The delay (in cycles) of switch allocation.

\item[vc\_allocator] The type of allocator used for virtual-channel
allocation.  See Section~\ref{sec:alloc} for a list of the possible
allocators.

\item[vc\_alloc\_delay] The delay (in cycles) of virtual-channel
allocation.

\end{opt_list}


\subsubsection{The NoC router}
\label{sec:noc_router}

The NoC router (\texttt{router = noc}) is a simplified variant of the input-queued router. It does not support input or output speedup. Also, it does not support multi-cycle allocation, but does support output-first allocation.


\subsubsection{The event-driven router}
\label{sec:event_router}

The event-driven router (\texttt{router = event}) is a
microarchitecture designed specifically to support a large number of
virtual channels (VCs) efficiently.  Instead of continuously polling
the state of the virtual channels, as in the input-queued router, only
changes in VC state are tracked.  The efficiency then comes from the
fact that the number of state changes per cycle is constant and
independent of the number of VCs.

\subsection{Allocators}
\label{sec:alloc}

Many of the allocators used in the simulator are configurable (see
the input-queued router in Section~\ref{sec:iq_router}) and several
allocation algorithms are available.
\begin{opt_list}{wavefront}

\item[max\_size] Maximum-size matching. 
\item[islip] iSLIP separable allocator.
\item[pim] Parallel iterative matching separable allocator.
\item[loa] Lonely output allocator.
\item[wavefront] Wavefront allocator.
\item[matrix] Separable input-first allocator built from matrix arbiters.
\item[round\_robin] Separable input-first allocator built from round-robin arbiters.
\item[select] Priority-based allocator.  Allocation is performed as in
iSLIP, but with preference towards higher priority packets (see
\texttt{priority} option in Section~\ref{sec:traffic}).

\end{opt_list}

Allocation can also be improved by performing multiple iterations of
the algorithm and the number of iterations is controlled by the
\texttt{alloc\_iters} parameter.

\subsection{Traffic}
\label{sec:traffic}

The rate at which flits are injected into the simulator is set using
the \texttt{injection\_rate} option.  The simulator's cycle time is a
flit cycle, the time it takes a single flit to be injected at a
source, and the injection rate is specified in flits per flit cycle.
For example, setting \texttt{injection\_rate = 0.25} means that each
source injects a new flit one of every four simulator cycles.  The
injection process can also be specified as either Bernoulli
(\texttt{injection\_process = bernoulli}) or an on-off process
(\texttt{injection\_process = on\_off}).  The burstiness of the latter
injection process is controlled via the \texttt{burst\_alpha} and
\texttt{burst\_beta} parameter.  See PPIN Section 24.2.2 for a
description of the on-off process and its parameters.

The unit of injection is packets, which may be comprised of many
flits.  The number of flits per packet is set using the
\texttt{const\_flits\_per\_packet} option.  Each packet may also have an
associated priority, either age-based (\texttt{age}) or none
(\texttt{none}), as specified by the \texttt{priority} option.

The simulator also supports several different traffic patterns that
are specified using the \texttt{traffic} option.  To describe these
patterns, we use the same notation of PPIN Section 3.2: $s_i$ ($d_i$)
denotes the $i^\textrm{th}$ bit of the source (destination) address
whereas $s_x$ ($d_x$) denotes the $x^\textrm{th}$ radix-$k$ digit of
the source (destination) address.  The bit length of an address is $b
= \log_2 N$, where $N$ is the number of nodes in the network.

\begin{opt_list}{transpose}
\item[uniform] Each source sends an equal amount of traffic to each
destination (\texttt{traffic = uniform}).
\item[bitcomp] Bit complement. $d_i = \neg s_i$.
\item[bitrev] Bit reverse. $d_i = s_{b-i-1}$.
\item[shuffle] $d_i = s_{i-1 \mod b}$.
\item[transpose] $d_i = s_{i+b/2 \mod b}$.
\item[tornado] $d_x = s_x + \lceil k/2 \rceil - 1 \mod k$.
\item[neighbor] $d_x = s_x + 1 \mod k$.
\item[randperm] Random permutation.  A fixed permutation traffic
pattern is chosen uniformly at random from the set of all
permutations.  The seed used to generate this permutation is set by
the \texttt{perm\_seed} option.  So, randomly selecting values for
\texttt{perm\_seed} gives a random sampling of permutation while a
fixed value of \texttt{perm\_seed} allows the same permutation to be
used for several experiments.
\end{opt_list}

\subsection{Simulation parameters}
\label{sec:sim_params}

The duration and other aspects of a simulation are controlled using
the set of simulation parameters.

\begin{opt_list}{warmup\_periods}

\item[sim\_type] A simulation can either focus on
\texttt{throughput} or \texttt{latency}.  The key difference between
these two types is that a \texttt{latency} simulation will wait for
all measurement packets to drain before ending the simulation to
ensure an accurate latency measurement.  In \texttt{throughput}
simulations, this final drain step is eliminated to allow simulation
of networks operating beyond their saturation point.

\item[sample\_period] The sample period is expressed in simulator
cycles and is used as a multiplier when specifying the warm-up length
of a simulation and the maximum number of samples.  Also, intermediate
statistics are displayed once every \texttt{sample\_period} cycles.

\item[warmup\_periods] The length of the simulator warm up expressed
as a multiple of the \texttt{sample\_period}.  After warming up, all
statistics counters are reset.

\item[max\_samples] The total length of simulation expressed as a
multiple of the \texttt{sample\_period}.

\item[latency\_thres] If the sampled latency of the current simulation
exceeds \texttt{latency\_thres}, the simulation is immediately ended.

\item[sim\_count] The number of back-to-back simulations to run for the
given configuration.  Useful for creating ensemble averages of
particular statistics.

\item[seed] A random seed for the simulation.

\item[reorder] A non-zero value indicates that packet order should be
maintained and reordering time is accounted for in the overall latency.

\end{opt_list}

\appendix

\section{Random number generation}

The simulator uses Knuth's integer and floating point pseudorandom
number generators.  These algorithms and their explanations appear in
``The Art of Computer Programming: Seminumerical Algorithms''.

\end{document}
